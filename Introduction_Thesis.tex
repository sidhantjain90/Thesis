\chapter{Introduction}
\section{General Introduction}
Plasmas exist in several forms in the Universe such as surface of the sun, interstellar gaseous clouds. They can also be observed near Earth's surface in the form of Lightning strikes, sprites and auroras. However, they do not exist naturally on Earth and hence created artificially by humans for various scientific and commercial purposes. A Plasma is formed when a medium (gas) is provided with sufficient energy (here electric field) to remove electrons from atoms of the medium. This energy may be provided by application of an electric field, joule heating or other sources such as cosmic rays. So, Plasmas primarily consists of electrons, positively charged ions and neutrals.

Plasmas can be classified into two types based on relative temperatures of electrons, ions and neutrals, namely, thermal and non-thermal plasmas. In thermal plasmas, the electrons are in thermal equilibrium with the positive ions. In non-thermal plasma, the temperature of the electron is much higher than than those of the ions and neutrals. This is because the energy transfer from applied electric field to electrons is much faster than the energy loss due to collisional energy transfer between electrons and ions or heavy species. This is due to the relatively smaller mass of the electrons. Non-Equilibrium Plasmas also referred as Cold Plasmas have several applications such as plasma-assisted combustion, air purification, ozone production, etc.

The phenomenon of Non-Equilibrium cold plasmas is studied with the help of a theory developed by Loeb and Meek \cite{Loeb and Meek , 1940a and 1940b} who described the formation of a highly non-linear space charge waves before the occurence of electrical breakdown, which creates an internal electric field of its own, comparable to the applied (external) field. These space charge waves are commonly known as streamers. Streamers are considered as the precursors to spark discharges, lightening\cite{Bazelyan et Raizer 2000} and sprites \cite{Franz et al 1990; Bourdon et al 2007}. They are able to initiate spark discharges in relatively shorter gaps i.e several centimetres, at near ground pressures in air.  

Numerical Modeling of streamers.
Several attempts were made in order to model analytically the various properties of the streamers. Dhali and Williams proposed a 2-D numerical model using continuty equations for electrons and positive ions coupled with Poisson's equation for Electric Field to describe the streamer propagation. A more detailed description can be found in their paper {/cite Dhalli 1987}.

Challenges occurring in numerically modeling streamers.


\section{Scope of the Ph.D thesis}




\chapter{Streamers' Theory}

\section{Important processes in discharges}
A gaseous medium such as air always contain some ions and electrons. This is due to natural processes such as cosmic radiation, photoionization, etc. When an electric field is applied to the gas, it accelerates these electrons and ions . As electrons move through the medium they collide with neutral atoms or molecules. Important collisions are:

1. Impact Ionization:

2. Recombination:

3. Excitation:

4. Attachment:



\section{Electron Avalanche}

If the energy acquired by a single electron is more than the ionization potential , it knocks off another electron from a neutral molecule. As a consequence, two slow electrons are accelerated further by the reigning electric field. Until the breakdown value of electric field has been achieved, there is a balance between the production of electrons (due to impact ionization) and loss of electrons (due to attachment). Above the breakdown field, the number of electrons start to increase exponentially giving rise to a phenomenon known as avalanche.

\section{Avalanche to Streamer Transition}


\section{Streamer Discharge Mechanism}

Streamers are thin ionized plasma channels formed between two oppositely charged electrodes when space charge field become comparable to the external electric field (Meek's criterion). They can propagate to either direction i.e towards anode or cathode as a result of distortion of the external electric field. They occur before the complete breakdown of a gas or dieclectric medium. Hence, they are also referred as precursors of electrical breakdown.  

\subsection{Challenges in Modeling Streamers}




\chapter{Numerical Models}

\section{Kinetic Models}

\section{Fluid Models}

\section{Hybrid Models}

\section{Solving Poisson's Equation}





\chapter{Photoionization Model}



\chapter{Grid Generation Methods}

\chapter{Parallelization Methods}
