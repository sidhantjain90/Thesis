\chapter{Introduction}
\section{General Introduction}
Plasmas exist in several forms in the Universe such as surface of the sun, interstellar gaseous clouds. They can also be observed near Earth's surface in the form of Lightning strikes, sprites and auroras. However, they do not exist naturally on Earth and hence created artificially by humans for various scientific and commercial purposes. A Plasma is formed when a medium (gas) is provided with sufficient energy (here electric field) to remove electrons from atoms of the medium. This energy may be provided by application of an electric field, joule heating or other sources such as cosmic rays. So, Plasmas primarily consists of electrons, positively charged ions and neutrals.

Plasmas can be classified into two types based on relative temperatures of electrons, ions and neutrals, namely, thermal and non-thermal plasmas. In thermal plasmas, the electrons are in thermal equilibrium with the positive ions. In non-thermal plasma, the temperature of the electron is much higher than than those of the ions and neutrals. This is because the energy transfer from applied electric field to electrons is much faster than the energy loss due to collisional energy transfer between electrons and ions or heavy species. This is due to the relatively smaller mass of the electrons. Non-Equilibrium Plasmas also referred as Cold Plasmas have several applications such as plasma-assisted combustion, air purification, ozone production, etc.

The phenomenon of Non-Equilibrium cold plasmas was studied with the help of a theory developed by Loeb and Meek \cite{Loeb and Meek , 1940a and 1940b} who described the formation of a highly non-linear space charge waves before the occurence of electrical breakdown, which creates an internal electric field of its own, comparable to the applied (external) field. These space charge waves are commonly known as streamers. Streamers are considered as the precursors to spark discharges, lightening\cite{Bazelyan et Raizer 2000} and sprites \cite{Franz et al 1990; Bourdon et al 2007}. They are able to initiate spark discharges in relatively shorter gaps i.e several centimetres, at near ground pressures in air.  

Several attempts were made in order to model analytically the various quantitative properties of the streamers \cite{Wright & London 1964, }. Mostly, the solutions were one-dimensional and they made unsatisfactory approximations for the solution of electric field. With the advent of more powerful computer and faster algorithms, numerical solution the streamer tranport equations was made possible. Earliest numerical modeling of streamers was done by Davies, Davies and Evans \cite{Davies et al 1971}, who developed a numerical algorithm for the treatment of hydronamic model representation of the streamer transport. They used a first-order method of characterstics for integrating the transport equations. However, the algorithm was for one dimensional treatment of transport and had stability issues when implemented for two-dimensional equations. Several teams \cite{Kline, 1974; Abbas & Bayle, 1980; Yoshida & Tagashira, 1976} adopted this method. Two main problems hampered the numerical solution of streamer propagation. First was that problem was atleast 2-D. Second was the steep gradients which are difficult to capture. Dhali and Williams proposed a numerical approach to the problem as a 2-D numerical model using drift-diffusion equations for electrons and positive ions coupled with Poisson's equation for Electric Field to describe the streamer propagation. They used a Flux-Corrected Transport technique to deal with the steep gradients. A more detailed description can be found in their paper \cite{Dhalli 1987}. The FCT technique was further adopted by Kushner's team \cite{Kushner et al 1989} to study the electron-velocity distribution (kinetic model). But solving Boltzmann equations using kinetic model proves to be very time consuming as it takes into account the solution calculation of several super particles using Lagrangian approach. FCT was also implemented by Morrow's team to study streamer propagation in air \cite{Morrow et al, 1997}. Several works were done for non-uniform gaps. \cite{Vitello et al, 1994} used a point-plane gap to perform fully 2-D simulation of streamer propagation. Babaeva and team \cite{Babaeva 1997} used a sphere-plane configuration. The computational grid is a tangent to the spherical electrode (i.e a single point of contact) and hence the electrode is not included in the domain.A similar approach was implemented by Luque's team \cite{Luque et al 2008} where they studied 2-D streamer propagation using point-plane geometry with one point of contact of electrode with the computational domain. Kulikovsky \cite{Kulikovsky 1994} and Pancheshnyi \cite{Pancheshnyi S 2005} used a hyberboloid-plane gap for streamer modeling.

Leer developed a new second-order accurate algorithm for integrating flow equations based on conservation laws named monotonic upwind-centered scheme for conservation-law (MUSCL) scheme \cite{Brahm Van Leer, 1979}. This scheme was implemented by Eichwald and team \cite{Eichwald et al, 1998, 2006} for 2-D modeling of streamer discharges in collaboration with the Finite Volume Methods. FVM was also adopted by Kushner and team \cite{Kushner et al 2003} who studied the breakdown processes in metal halid lamps. Zakari and team \cite{Zakari et al 2015} used median-dual control volumes for FVM with unstructured grid for studying the streamer discharges between point-plane electrodes.

In early 2000s, Gheorghiou and team introduced finite element methods to solve the electro-hydrodynamic transport equations of streamer propagation on plane-plane geometry \cite{Gheorghiou et al, 1999}. They combined the FEM technique with FCT to handle accurately the steep density gradients. A critical analysis for the use of FEM-FCT on unstructured and FVM-MUSCL on structured methods in the streamer modeling was done by Ducasse and team \cite{Ducasse et al 2007}. According to their analysis, both FEM-FCT and FVM-MUSCL methods gave numerically accurate and close solutions. But the implementation of Finite Elements is much more complex and the also the object oriented programming approach with FEM increases the memory needs. Also, the computational time for FEM is much more than FVM, even though it uses less number of nodes. However, the use of FVM methods is restricted to complex geometries. To overcome this restriction, Celestin and team \cite{Celestin et al, 2009} developed a method named Ghost Fluid Method which takes into account the geometry of the electrodes to accurately solve the Poisson's equation and subsequently the evolution of the charge densities. To solve the poisson's equation more accurately , Kacem & team \cite{Kacem et al 2011} developed a Full MultiGrid (FMG) method to solve the linear matrix systems by smoothening the solution on the set of progressively coarsened grids.

Dhalli's team \cite{Dhalli & Williams, 1987} took into account the process of photoionization by adding a photoionization source term in the transport equation. There have been advancements in the calculation of photoionization source term using integral models with coefficients obtained from Penney and Hummert's experiments \cite{Penney et al , 1970} or model proposed by Zheleznyak \cite{Zheleznyak, 1982}.  

\section{Scope of the Ph.D thesis}




\chapter{Streamers' Theory}

\section{Important processes in discharges}
A gaseous medium such as air always contain some ions and electrons. This is due to natural processes such as cosmic radiation, photoionization, etc. When an electric field is applied to the gas, it accelerates these electrons and ions . As electrons move through the medium they collide with neutral atoms or molecules. Important collisions are:

1. Impact Ionization:

2. Recombination:

3. Excitation:

4. Attachment:



\section{Electron Avalanche}

If the energy acquired by a single electron is more than the ionization potential , it knocks off another electron from a neutral molecule. As a consequence, two slow electrons are accelerated further by the reigning electric field. Until the breakdown value of electric field has been achieved, there is a balance between the production of electrons (due to impact ionization) and loss of electrons (due to attachment). Above the breakdown field, the number of electrons start to increase exponentially giving rise to a phenomenon known as avalanche.

\section{Avalanche to Streamer Transition}


\section{Streamer Discharge Mechanism}

Streamers are thin ionized plasma channels formed between two oppositely charged electrodes when space charge field become comparable to the external electric field (Meek's criterion). They can propagate to either direction i.e towards anode or cathode as a result of distortion of the external electric field. They occur before the complete breakdown of a gas or dieclectric medium. Hence, they are also referred as precursors of electrical breakdown.  

\subsection{Challenges in Modeling Streamers}




\chapter{Numerical Models}

\section{Kinetic Models}

\section{Fluid Models}

\section{Hybrid Models}

\section{Solving Poisson's Equation}

\subsection{Local Field Approximation}

Kundhardt and team developed the local field approximation. \cite{Kunhardt et al 1988}.





\chapter{Photoionization Model}



\chapter{Grid Generation Methods}

\chapter{Parallelization Methods}
